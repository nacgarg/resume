\documentclass[a4paper]{article}
\usepackage{url}
\usepackage{hyperref}
\usepackage[margin=1in]{geometry}
\usepackage{datetime}


\newcommand{\lineunder} {
    \vspace*{-8pt} \\
    \hspace*{-18pt} \hrulefill \\
}

\newcommand{\header} [1] {
    {\hspace*{-18pt}\vspace*{6pt} \textsc{#1}}
    \vspace*{-6pt} \lineunder
}

\begin{document}
\vspace*{-40pt}
    

% \vspace*{-30pt}
\begin{center}
	{\Huge \scshape {Nachiketa Gargi}}\\
	ngargi@umich.edu $\cdot$ github.com/nacgarg $\cdot$ (650) 335-8753 $\cdot$ Sunnyvale, CA \\
\end{center}
\header{Education}

\noindent
\textbf{University of Michigan} \hfill Aug 2018 - Apr 2022 (expected)
\vspace{-2mm}
\begin{itemize}
    \setlength{\itemindent}{-3mm}
    \item[] \textit{B.S. in Computer Science (expected)}\vspace{-3mm}
    \item[] \textit{B.F.A. in Performing Arts Technology (expected)}
\end{itemize}
\noindent
\textbf{The Nueva School} \hfill Aug 2014 - June 2018\\
\vspace{-4mm}

FRC Robotics Team 4904, Jazz Ensemble, Programming Club, Music Production and DJ Club\\


\noindent
\header{Work Experience}

\noindent
\textbf{MIDAS Music Theory} - \url{midasmusictheory.org} \hfill Sep 2018 - Present\\
\textit{Research Assistant} \\ 
\vspace{-6mm}
\begin{itemize}
    \item Working on developing a method to transcribe a corpus of Indian music compositions automatically
\end{itemize}

\noindent
\textbf{YouSound} - \url{yousound.com} \hfill June 2017 - Sep 2018\\
\textit{Full Stack Developer} \\
\vspace{-6mm}
\begin{itemize} \itemsep 0.1pt
	\item Created a scaleable chat platform to accompany the rest of the site using Node and socket.io, deployed on AWS.
	\item Researched and documented setup instructions and assisted in implementation for a web-scale video live-streaming platform (like twitch.tv) using AWS Elemental MediaLive, MediaPackage, and CloudFront.
\end{itemize}

\noindent
\textbf{Primity Bio} - \url{primitybio.com} \hfill June 2016 - Sep 2016\\
\textit{Software Engineering Intern} \\
\vspace{-6mm}
\begin{itemize} \itemsep 0.1pt
	\item Worked on a web-based realtime, collaborative data analysis platform for clients using test-based development with Node, Angular, and MongoDB.
	\item This work was presented at an FDA conference in Washington, D.C.
\end{itemize}



\header{Projects}
\noindent
\textbf{ISMIR 2018} \hfill Jun 2018 - Sep 2018\\
\textit{Adversarial Reinforcement Learning for Music Generation} \hfill Python, Keras, \LaTeX\\
\vspace{-25pt}
\begin{paragraph}{}
Wrote paper describing my findings when experimenting with various methods of music generation. The paper was accepted as part of the 2018 Late-Breaking Demos session at ISMIR. Available \href{http://ismir2018.ircam.fr/pages/events-lbd.html}{on \underline{ISMIR's website}}\\
\end{paragraph}

\noindent
\textbf{FieldAC} \hfill Jan 2018 - Apr 2018\\
\textit{FRC Robot and Object Localization} \hfill C++, OpenCV, Darknet\\
\vspace{-25pt}
\begin{paragraph}{}
As part of FRC team, trained a custom model for an object detection framework (YOLOv3) on game pieces. Created field model that used optical flow in conjunction with YOLOv3, onboard LiDAR, and IMU to estimate pose of robot and game pieces on the field. Model was used for an autonomous routine to manipulate the nearest game piece.\\

\end{paragraph}
\noindent
\textbf{AutoMuse} \hfill Oct 2014 - June 2016\\
\textit{Automated computer music composition} \hfill Python, Keras
\vspace{-13pt}
\begin{paragraph}{}
Used markov chains and LSTM neural networks to generate music from a dataset of scraped MIDI files. \\ 

\end{paragraph}

\noindent
\header{Skills}

\noindent
\begin{tabular}{ l l }
	Programming Languages & JS, Python, Go, Bash, C++, Java, Max/MSP \\
	Creative Software             & Ableton Live, After Effects, Photoshop, Logic    \\
	Other                  & Markdown, AWS, GCP, Git, Slack, Trello        \\
\end{tabular}
\vspace{8mm}
\pagenumbering{gobble}
\noindent
\begin{center}
    \sc{Nachiketa Gargi, \monthname{} \the\year{}}
\end{center}
\ 
\end{document}
