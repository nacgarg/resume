\documentclass[11pt]{extarticle}
\usepackage{url}
\usepackage{hyperref}
\usepackage[margin=1in]{geometry}
\usepackage{datetime}


\newcommand{\lineunder} {
	\vspace*{-8pt} \\
	\hspace*{-18pt} \hrulefill \\
}

\newcommand{\header} [1] {
	{\hspace*{-18pt}\vspace*{6pt} \textsc{#1}}
	\vspace*{-6pt} \lineunder
}

\begin{document}
\vspace*{-30pt}
	

% \vspace*{-30pt}
\begin{center}
	{\Huge \scshape {Nachiketa Gargi}}\\
	ngargi@umich.edu $\cdot$ \url{github.com/nacgarg} $\cdot$ (650) 335-8753 $\cdot$ Sunnyvale, CA \\
\end{center}
\noindent
\header{Education}
\noindent
\textbf{University of Michigan} \hfill Class of 2022
\vspace{-2mm}
\begin{itemize}
	\setlength{\itemindent}{-3mm}
	\item[] \textit{B.S.E. in Computer Science (expected)}\vspace{-3mm}
	\item[] \textit{B.F.A. in Performing Arts Technology (expected)}
	\item[] Relevant Coursework: \\ Programming and Introductory Data Structures (\textsc{eecs 280}), Calculus II (\textsc{math 116}) at U-M \\ Advanced Machine Learning, Differential Equations, Advanced Mechanics in high school
	\item[] Clubs and Organizations:\\ \href{https://umautonomy.com/}{UM::Autonomy} (autonomous vehicle design team), \href{http://www.midasmusictheory.org/}{MIDAS} (Michigan Institute of Data Science) Music Theory, \href{https://michiganprojectmusic.github.io/index.html}{Project Music} (experimental musical instrument design) 
\end{itemize}

\noindent
\header{Work Experience}
\noindent
\textbf{MIDAS Music Theory} - \url{midasmusictheory.org} \hfill Sep 2018 - Present\\
\textit{Research Assistant} \\ 
\vspace{-6mm}
\begin{itemize} \itemsep 0.1pt
	\item Member of the Michigan Institute for Data Science project “A Computational Study of Patterned Melodic Structures Across Musical Cultures,” a collaborative research project between EECS, Math, and Music faculty.
	\item Developing automatic methods to transcribe scanned images of Indian classical music compositions.
\end{itemize}

\noindent
\textbf{YouSound} - \url{yousound.com} \hfill Jun 2017 - Sep 2018\\
\textit{Full Stack Developer} \\
\vspace{-6mm}
\begin{itemize} \itemsep 0.1pt
	\item Created a scaleable chat platform to accompany the rest of the site using Node and socket.io, deployed on AWS.
	\item Researched and documented setup instructions and assisted in implementation for a web-scale video live-streaming platform (like twitch.tv) using AWS Elemental MediaLive, MediaPackage, and CloudFront.
\end{itemize}

\noindent
\textbf{Primity Bio} - \url{primitybio.com} \hfill Jun 2016 - Sep 2016\\
\textit{Software Engineering Intern} \\
\vspace{-6mm}
\begin{itemize} \itemsep 0.1pt
	\item Worked on a web-based realtime, collaborative data analysis platform for clients using test-based development with Node, Angular, and MongoDB.
	\item This work was presented at an FDA conference in Washington, D.C.
\end{itemize}

\newpage
\noindent
\header{Projects and Publications}
\noindent
\href{https://github.com/nacgarg/music-makeathon}{\textbf{Music Makeathon}} \hfill Oct 2018\\
\textit{Realtime Sound-controlled Audio and Video Resampling} \hfill C++, JUCE, Max \\
\vspace{-25pt}
\begin{paragraph}{}
	Won first place in the Project Music 2018 Makeathon with a live, realtime remixer that uses FFT to replace input audio with audio from existing songs. In addition, used Max to control video playback based on frequency and amplitude of input audio. The entire project was completed within 18 hours.\\
\end{paragraph}


\noindent
\href{http://ismir2018.ircam.fr/pages/events-lbd.html}{\textbf{ISMIR 2018}} \hfill Jun 2018 - Sep 2018\\
\textit{Adversarial Reinforcement Learning for Music Generation} \hfill Python, Keras, \LaTeX\\
\vspace{-25pt}
\begin{paragraph}{}
\begin{sloppypar}
Using a generative adversarial network with music theory constraints. \href{http://ismir2018.ircam.fr/pages/events-lbd.html}{ISMIR 2018, Late Breaking Session}.\\
\end{sloppypar}
\end{paragraph}

\noindent
\href{https://github.com/RoboticsTeam4904/FieldAC}{\textbf{FieldAC}} \hfill Jan 2018 - Apr 2018\\
\textit{FRC Robot and Object Localization} \hfill C++, OpenCV, Darknet\\
\vspace{-25pt}
\begin{paragraph}{}
As part of FRC team, trained a custom model for an object detection framework (YOLOv3) on game pieces. Created field model that used optical flow in conjunction with YOLOv3, onboard LiDAR, and IMU to estimate pose of robot and game pieces on the field. Model was used for an autonomous routine to manipulate the nearest game piece.\\
\end{paragraph}

\noindent
\href{https://kanooli.com/}{\textbf{Kanooli}} \hfill Oct 2016 - Present\\
\textit{Electronic Music Production} \hfill Ableton Live\\
\vspace{-25pt}
\begin{paragraph}{}
	Regularly produce electronic compositions under the alias Kanooli. Developed branding strategy and accompanying website, and have accumulated over 100k plays through releases on several labels.\\
\end{paragraph}

\noindent
\textbf{Schedulizer} \hfill Aug 2017 - Jun 2018\\
\textit{School Assistant Bot for Telegram} \hfill Go, Telegram Bot API \\ 
\vspace{-25pt}
\begin{paragraph}{}
Developed Telegram chat bot used by the majority of students at my high school to assist students with the frequently changing class schedule and upcoming homework assignments. \\
\end{paragraph}

\noindent
\href{https://github.com/nacgarg/AutoMuse}{\textbf{AutoMuse}} \hfill Oct 2014 - June 2016\\
\textit{Automated computer music composition} \hfill Python, Keras \\
\vspace{-25pt}
\begin{paragraph}{}
Used markov chains and LSTM neural networks to generate music from a dataset of scraped MIDI files. \\ 

\end{paragraph}

\noindent
\header{Skills}
\noindent

\begin{tabular}{ l l }
	Programming Languages & JS, Python, Go, Bash, C++, Java, Max/MSP \\
	Creative Software             & Ableton Live, After Effects, Photoshop, Logic    \\
	Other                  & Git, AWS, GCP, Slack, Trello        \\
\end{tabular}
\vspace{8mm}
% \pagenumbering{gobble}
\noindent
\begin{center}
	\sc{Nachiketa Gargi, \monthname{} \the\year{}}
\end{center}
\end{document}
